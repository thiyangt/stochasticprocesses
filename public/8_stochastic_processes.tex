\PassOptionsToPackage{unicode=true}{hyperref} % options for packages loaded elsewhere
\PassOptionsToPackage{hyphens}{url}
\documentclass[12pt,ignorenonframetext,]{beamer}
\IfFileExists{pgfpages.sty}{\usepackage{pgfpages}}{}
\setbeamertemplate{caption}[numbered]
\setbeamertemplate{caption label separator}{: }
\setbeamercolor{caption name}{fg=normal text.fg}
\beamertemplatenavigationsymbolsempty
\usepackage{lmodern}
\usepackage{amssymb,amsmath}
\usepackage{ifxetex,ifluatex}
\usepackage{fixltx2e} % provides \textsubscript
\ifnum 0\ifxetex 1\fi\ifluatex 1\fi=0 % if pdftex
  \usepackage[T1]{fontenc}
  \usepackage[utf8]{inputenc}
\else % if luatex or xelatex
  \ifxetex
    \usepackage{mathspec}
  \else
    \usepackage{fontspec}
\fi
\defaultfontfeatures{Ligatures=TeX,Scale=MatchLowercase}







\fi

  \usetheme[]{metropolis}






% use upquote if available, for straight quotes in verbatim environments
\IfFileExists{upquote.sty}{\usepackage{upquote}}{}
% use microtype if available
\IfFileExists{microtype.sty}{%
  \usepackage{microtype}
  \UseMicrotypeSet[protrusion]{basicmath} % disable protrusion for tt fonts
}{}


\newif\ifbibliography


\hypersetup{
      pdftitle={STA 331 2.0 Stochastic Processes},
        pdfauthor={Dr Thiyanga S. Talagala},
          pdfborder={0 0 0},
    breaklinks=true}
%\urlstyle{same}  % Use monospace font for urls





  \usepackage{longtable,booktabs}
  \usepackage{caption}
  % These lines are needed to make table captions work with longtable:
  \makeatletter
  \def\fnum@table{\tablename~\thetable}
  \makeatother


% Prevent slide breaks in the middle of a paragraph:
\widowpenalties 1 10000
\raggedbottom

  \AtBeginPart{
    \let\insertpartnumber\relax
    \let\partname\relax
    \frame{\partpage}
  }
  \AtBeginSection{
    \ifbibliography
    \else
      \let\insertsectionnumber\relax
      \let\sectionname\relax
      \frame{\sectionpage}
    \fi
  }
  \AtBeginSubsection{
    \let\insertsubsectionnumber\relax
    \let\subsectionname\relax
    \frame{\subsectionpage}
  }



\setlength{\parindent}{0pt}
\setlength{\parskip}{6pt plus 2pt minus 1pt}
\setlength{\emergencystretch}{3em}  % prevent overfull lines
\providecommand{\tightlist}{%
  \setlength{\itemsep}{0pt}\setlength{\parskip}{0pt}}

  \setcounter{secnumdepth}{0}


  \usepackage{amsfonts}
  \usepackage{mathrsfs}

  \title[]{STA 331 2.0 Stochastic Processes}

  \subtitle{8. Birth and Death Processes}

  \author[
        Dr Thiyanga S. Talagala
    ]{Dr Thiyanga S. Talagala}

  \institute[
    ]{
    Department of Statistics, University of Sri Jayewardenepura
    }

\date[
      
  ]{
    }


\begin{document}

% Hide progress bar and footline on titlepage
  \begin{frame}[plain]
  \titlepage
  \end{frame}



\begin{frame}{Birth and Death Processes}
\protect\hypertarget{birth-and-death-processes}{}
\begin{itemize}
\item
  The birth-and-death process is a subclass of \textbf{continuous-time
  Markov chains}.
\item
  The birth-and-death processes are characterized by the property that
  whenever a transition occurs from one state to another, then this
  transition can be to a neighbouring state only.
\end{itemize}
\end{frame}

\begin{frame}{Transition types}
\protect\hypertarget{transition-types}{}
\begin{itemize}
\item
  a transition occurs from one state to another and this transition can
  be to a neighbouring state only.

  \begin{itemize}
  \item
    Eg: State space \(S = \{0, 1, 2, ..., i, ...\}\)
  \item
    transition that occurs from state \(i\), can be only to a
    neighboring state \((i-1)\) or \((i+1)\).
  \end{itemize}
\end{itemize}
\end{frame}

\begin{frame}{Birth rate and Death rate}
\protect\hypertarget{birth-rate-and-death-rate}{}
\begin{block}{Birth rate}
\protect\hypertarget{birth-rate}{}
\(\lambda_i\) - birth rate from state \(i\), \(i \geq 0\)
\end{block}

\begin{block}{Death rate}
\protect\hypertarget{death-rate}{}
\(\mu_i\) - death rate from state \(i\), \(i \geq 0\)
\end{block}
\end{frame}

\begin{frame}{Queueing systems}
\protect\hypertarget{queueing-systems}{}
\begin{enumerate}
\item
  Birth - equivalent to the arrival of a customer.
\item
  Death - equivalent to the departure of a served customer.
\end{enumerate}
\end{frame}

\begin{frame}{Notations}
\protect\hypertarget{notations}{}
A continuous-time Markov chain \([X(t), t \in T]\) with state space
\(S = \{0, 1, 2, ...\}\) with rates

\[q_{i, i+1}=\lambda_i, \text{ } i= 0, 1, ...,\]

\[q_{i, i-1}=\mu_i, \text{ } i= 1, 2, ...,\]

\[q_{i, j}=0, \text{ } j \neq i \pm 1, \text{ } j \neq i, \text{ } i=0, 1, ..., \text{ and}\]

\[q_i = (\lambda_i + \mu_i), \text{ } i=0, 1, ..., \text{ and } \mu_0 = 0.\]
\end{frame}

\begin{frame}{Pure birth process, pure death process, birth-and-death
process}
\protect\hypertarget{pure-birth-process-pure-death-process-birth-and-death-process}{}
\begin{enumerate}
[i)]
\item
  a pure birth process if \(\mu_i = 0\) for \(i=1, 2, ...\)

  \begin{itemize}
  \tightlist
  \item
    No decrements, only increments.
  \end{itemize}
\item
  a pure death process if \(\lambda_i = 0\) for \(i=1, 2, ...\)

  \begin{itemize}
  \tightlist
  \item
    No increments, only decrements.
  \end{itemize}
\item
  a birth-and-death process if some of the \(\lambda_i\)'s and some of
  the \(\mu_i\)'s are positive.
\end{enumerate}
\end{frame}

\begin{frame}{Examples of random phenomena modelled through birth and
death processes}
\protect\hypertarget{examples-of-random-phenomena-modelled-through-birth-and-death-processes}{}
\begin{itemize}
\item
  Spread of epidemic disease
\item
  Mutant gene dynamics
\item
  Cell kinetics (proliferation of cancer cells)
\end{itemize}
\end{frame}

\begin{frame}{Special cases}
\protect\hypertarget{special-cases}{}
\begin{enumerate}
\item
  Linear birth process: Yule-Furry process
\item
  Linear death process
\item
  Linear birth and death process
\item
  M/M/I queue
\end{enumerate}
\end{frame}

\begin{frame}{Pure Birth Process}
\protect\hypertarget{pure-birth-process}{}
\begin{itemize}
\item
  Special case of a \textbf{continuous-time Markov process} and a
  \textbf{generalisation of a Poisson process}.
\item
  Consider a population of individuals where only the appearances of new
  individuals, which are called ``birth'' occur.
\end{itemize}
\end{frame}

\begin{frame}{General birth processes}
\protect\hypertarget{general-birth-processes}{}
Let us consider a birth process whose total number of individuals at
time \(t\) is denoted by a discrete random variable \(N(t)\). As
parameter \(t\) varies \(\{ N(t): t\geq 0\}\) represents a stochastic
process with a continuous parameter (time) space and a discrete state
space.

Let us assume that the birth rate depends on the present size of the
population. Further we assume that the births occur according to the
following postulates:

\begin{math}
\tiny
 P[N(t+h) = n+k|N(t)=n]=\left\{
    \begin{array}{ll}
      \lambda_n h + o(h), & \mbox{k=1}\\
      o(h), & k \geq 2 \\
      1-\lambda_n h + o(h), & k = 0 \\
    \end{array}
  \right.
\end{math}
\end{frame}

\begin{frame}{General birth processes (cont)}
\protect\hypertarget{general-birth-processes-cont}{}
\textbf{Condition 1}

\begin{math}
\tiny
 P[N(t+h) = n+k|N(t)=n]=\left\{
    \begin{array}{ll}
      \lambda_n h + o(h), & \mbox{k=1}\\
      o(h), & k \geq 2 \\
      1-\lambda_n h + o(h), & k = 0 \\
    \end{array}
  \right.
\end{math}

where \(\lambda_n\) is the rate at which the births occur at time \(t\)
and \(n\) being the size of the population at time \(t\).

\textbf{Condition 2}

\(N(0) > 0\)
\end{frame}

\begin{frame}{Your turn}
\protect\hypertarget{your-turn}{}
Compare the differences in conditions between Poisson process,
Non-homogeneous Poisson Process and Birth Process
\end{frame}

\begin{frame}{Goal: Probability Mass Function of \(N(t)\)}
\protect\hypertarget{goal-probability-mass-function-of-nt}{}
What is the probability that the population size at a given time, t,
equals \(N(t) = n\)?

\[P_n(t) = P[N(t)=n]= ?\] For example,

\[P_0(t) = P[N(t)=0]= ?\]

\[P_1(t) = P[N(t)=1]= ?\]

\[P_2(t) = P[N(t)=2]= ?\]

\[.\]

\[.\] and so on..

In general

\[P_n(t) = P[N(t)=n]= ?\]
\end{frame}

\begin{frame}{Linear Birth Process (Yule-Furry Process)}
\protect\hypertarget{linear-birth-process-yule-furry-process}{}
When, \(\lambda_n = n \lambda\), i.e.~when the birth rate is linear in
the present size of the population.

Then the pure birth process is said to a \textbf{Linear Birth Process}
or \textbf{Yule-Furry Process}.

Let is assume that \textbf{there is only one individual in the
population initially, \(N(0) = 1\)}. It can be shown that for any
\(t > 0\).

\[P(N(t)=0)=0\]

\[P(N(t)=n)=e^{-\lambda t}(1-e^{-\lambda t})^{n-1}\text{ }, n \geq 1.\]
\end{frame}

\begin{frame}{Proof (general situation):}
\protect\hypertarget{proof-general-situation}{}
For \(n=0\)

\(P_0(t+h)=P(N(t)=0)P(N(t+h)=0|N(t)=0)\)

\(P_0(t+h)=P_0(t)(1-\lambda_0 h + o(h))\)

i.e.

\(P_0(t+h)=P_0(t)-\lambda_0 h P_0(t) + o(h)P_0(t)\)

\(lim_{h \to 0}\frac{P_0(t+h)-P_0(t)}{h}= -lim_{h \to 0}\lambda_0 P_0(t) + lim_{h \to 0}\frac{o(h)}{h}P_0(t)\)

i.e.

\(P_0'(t)=-\lambda_0P_0(t).\)

We assume that \textbf{there is only one individual in the population
initially, \(N(0) = 1\)}. Hence, \(P[N(t) = 0] = 0\). That is
\(P_0(t)=0\).
\end{frame}

\begin{frame}{Proof: (cont)}
\protect\hypertarget{proof-cont}{}
For \(n \geq 1\)

\begin{align*}\label{eq:pareto mle1}
P_n(t+h) &=   P(N(t)=n)P(N(t+h)=n|N(t)=n) + \\
&P(N(t)=n-1)P(N(t+h)=n|N(t)=n-1) + \\
&\sum_{r=2}^{n-1}P(N(t)=n-r)P(N(t+h)=n|N(t)=n-r)
\end{align*}

i.e

\begin{align*}
P_n(t+h) &= P_n(t) (1-\lambda_n h + o(h))+\\
&P_{n-1}(t)(\lambda_{n-1} h + o(h))+ \\
& o(h)
\end{align*}
\end{frame}

\begin{frame}{Proof: (cont)}
\protect\hypertarget{proof-cont-1}{}
\(P_n(t+h)= P_n(t) -\lambda_n h P_n(t) + \lambda_{n-1}hP_{n-1}(t) + o(h) \text{ for } n\geq 1\)

\(lim_{h \to 0}\frac{P_n(t+h)-P_n(t)}{h}= -\lambda_n P_n(t) + \lambda_{n-1} P_{n-1}(t)+ lim_{h \to 0}\frac{o(h)}{h}\)

i.e.

\(P'_n(t) = -\lambda_n P_n(t) + \lambda_{n-1}P_{n-1}(t) \text{ for } n\geq 1.\)

Therefore the partial differential-difference equations is

For \(n \geq 1\),
\(P'_n(t) = -\lambda_n P_n(t) + \lambda_{n-1}P_{n-1}(t).\)
\end{frame}

\begin{frame}
When \(n=1\)

\[P_1'(t) = -\lambda_1P_1(t),\]
\[\int\frac{P_1'(t)}{P_1(t)}dt=-\lambda_1\int dt,\]

\[ln P_1(t) = -\lambda_1t + c\] \[P_1(t) = c_1e^{-\lambda_1 t}\] When
\(t=0\), \(c_1=1\)

\[P_1(t) = e^{-\lambda_1 t}\]
\end{frame}

\begin{frame}
When \(n=2\)

\[P_2'(t) = -\lambda_2P_2(t) + \lambda_1P_1(t),\]

\[P_2'(t)+\lambda_2P_2(t) = \lambda_1 e^{-\lambda_1 t},\]

Multiply by \(e^{\lambda_2t}\)

\[P_2'(t) e^{\lambda_2t}+\lambda_2P_2(t) e^{\lambda_2t} = \lambda_1 e^{-\lambda_1 t}e^{\lambda_2t},\]

\[\int \frac{d}{dt}[e^{\lambda_2 t} P_2(t)]dt= \int \lambda_1 e^{(\lambda_2 - \lambda_1)t}dt,\]

\[e^{\lambda_2 t}P_2(t) = \frac{\lambda_1 e^{(\lambda_2 - \lambda_1)t}}{\lambda_2 - \lambda_1} + c\]
\end{frame}

\begin{frame}
When \(t=0\),

We know that \(P_2(0) = 0\). hence,

\(c=-\frac{\lambda_1}{\lambda_2 - \lambda_1}.\)

Hence,

\[P_2(t) = \frac{\lambda_1}{\lambda_2 - \lambda_1}e^{-\lambda_2t}[e^{(\lambda_2 - \lambda_1)t}-1]\]
\end{frame}

\begin{frame}{Linear birth process (Yule-Furry Process)}
\protect\hypertarget{linear-birth-process-yule-furry-process-1}{}
When,

\[\lambda_n = n\lambda.\]

That is the birth rate is linear in the present size of the population.

Let us assume that there is \textbf{only one individual in the
population} initially. That is \(N(0)=1\).

Then the difference-differential equations of the linear birth process
takes the form

\(P_n'(t) = -n\lambda P_n(t) + (n-1) \lambda P_{n-1}(t)\) for
\(n \geq 1\) with the initial conditions \(P_1(0)=1\) and \(P_n(0) = 0\)
for \(n \geq 2\).
\end{frame}

\begin{frame}{Linear birth process (Yule-Furry Process) (cont)}
\protect\hypertarget{linear-birth-process-yule-furry-process-cont}{}
\(P_n'(t) = -n\lambda P_n(t) + (n-1) \lambda P_{n-1}(t)\) for
\(n \geq 1\) with the initial conditions \(P_1(0)=1\) and \(P_n(0) = 0\)
for \(n \geq 2\).

Multiplying the equation for \(n\) by \(z^n\) and summing over all \(n\)
we obtain

\[\frac{\partial}{\partial t}\sum_{n=1}^{\infty}P_n(t)z^n=-\lambda z \frac{\partial}{ \partial z} \sum_{n=1}^{\infty}P_n(t) z^n + \lambda z^2 \frac{\partial}{\partial z} \sum_{n=1}^{\infty}P_{n-1}(t)z^{n-1}\]

Let \(\prod(z, t) = \sum_{n=1}^{\infty}P_n(t)z^n\). Then the above
equations becomes

\[\frac{\partial \prod (z, t)}{\partial t}=-\lambda z \frac{\partial \prod (z, t)}{\partial z} + \lambda z^2 \frac{\partial \prod (z, t)}{\partial z}\]
\end{frame}

\begin{frame}{Linear birth process (Yule-Furry Process) (cont)}
\protect\hypertarget{linear-birth-process-yule-furry-process-cont-1}{}
i.e.~\(\frac{\partial \prod (z, t)}{\partial t} = \lambda z (z-1) \frac{\partial \prod (z, t)}{\partial z}\)

\(\frac{\partial \prod (z, t)}{\partial t} - \lambda z (z-1) \frac{\partial \prod (z, t)}{\partial z} =0\)

Subsidiary equations take the form

\[\frac{dt}{1} = \frac{dz}{-\lambda z (z-1)} = \frac{d\prod}{0} \]

Two independent solutions can be obtained one from \(d \prod = 0\) and
the other from \(-\lambda dt = \frac{dz}{z(z-1)}.\)

\(d \prod =0 \Rightarrow \prod(z, t) = constant.\)

\(-\lambda dt = \frac{dz}{z(z-1)} \Rightarrow \frac{z}{z-1}e^{-\lambda t} = constant.\)
\end{frame}

\begin{frame}{Linear birth process (Yule-Furry Process) (cont)}
\protect\hypertarget{linear-birth-process-yule-furry-process-cont-2}{}
The general solution can be written as

\(\prod (z, t) = f\left(\frac{z}{z-1}e^{-\lambda t}\right)\) where \(f\)
is an arbitrary function.

The initial conditions \(P_1(0) =1\) and \(P_n(0)=0\) for \(n \geq 2\)
imply that \(\prod (z, 0) = z.\)

\[  \therefore \prod (z, 0) = f \left (\frac{z}{z-1} \right) = z.\]

Let \(\omega = \frac{z}{z-1}\) \(\Rightarrow\)
\(z= \frac{\omega}{\omega - 1}\) and hence we obtain
\(f(\omega) = \frac{\omega}{\omega -1 }\).
\end{frame}

\begin{frame}{Linear birth process (Yule-Furry Process) (cont)}
\protect\hypertarget{linear-birth-process-yule-furry-process-cont-3}{}
\[ \therefore \prod (z, t) = \frac{\frac{z}{z-1}e^{-\lambda t}}{\frac{z}{z-1}e^{-\lambda t} - 1} = \frac{z e^{-\lambda t}}{z e ^{-\lambda t} - (z-1)}= \left(1-\frac{z-1}{z}e^{-\lambda t} \right)^{-1}\]

Considering coefficients of \(z^n\) we have

\[P_n(t) = e^{-\lambda t}{(1-e^{-\lambda t})^{n-1}} \text{ for } n \geq 1.\]

In proving the above results we assume that initially there is only one
individual in the population. That is N(0)=1.

Now let's prove for the case \(N(0) = a, a \geq 1.\) For that we use
moment generating functions.
\end{frame}

\begin{frame}{Moment generating function of \(N(t)\)}
\protect\hypertarget{moment-generating-function-of-nt}{}
Let

\[M_{N(t)}(\theta, t) = E[e^{N(t)\theta}],\]

be the moment generating function of \(N(t).\) Then, for \(t > 0,\)

\begin{equation} \label{eq1}
\begin{split}
M_{N(t)}(\theta, t) & = \sum_{n=0}^\infty e^{n \theta} P(N(t)=n)\\
 & = \sum_{n=0}^\infty e^{n \theta} P_n(t).
\end{split}
\end{equation}
\end{frame}

\begin{frame}{Moment generating function of \(N(t)\) (cont.)}
\protect\hypertarget{moment-generating-function-of-nt-cont.}{}
We assume that \(N(0) = a > 0\). Hence, \(P_n(t) = 0\) for all
\(n < a\). Hence,

\begin{equation} \label{eq1}
\begin{split}
M_{N(t)}(\theta, t) = \sum_{n=a}^\infty e^{n \theta} P_n(t).
\end{split}
\end{equation}
\end{frame}

\begin{frame}{Moment generating function of \(N(t)\) (cont.)}
\protect\hypertarget{moment-generating-function-of-nt-cont.-1}{}
Now we take derivative w.r.t \(\theta\). Then we get,

\[\frac{\partial}{\partial \theta}M_{N(t)}(\theta, t) = \sum_{n=a}^\infty ne^{n \theta} P_n(t).\]

The derivative w.r.t \(t\) is

\begin{equation} \label{eq2}
\begin{split}
\frac{\partial}{\partial t}M_{N(t)}(\theta, t) &= \sum_{n=a}^\infty e^{n \theta} P'_n(t) \\
&= \sum_{n=a}^{\infty}e^{n \theta}[-n \lambda P_n(t) + (n-1) \lambda P_{n-1}(t)] \\
&= - \sum_{n=a}^{\infty} n e^{n \theta} \lambda P_n(t) + \sum_{n=a}^{\infty}(n-1)e^{n \theta}\lambda P_{n-1}(t)
\end{split}
\end{equation}
\end{frame}

\begin{frame}{Moment generating function of \(N(t)\) (cont.)}
\protect\hypertarget{moment-generating-function-of-nt-cont.-2}{}
Since \(P_{a-1}(t)=0\), the second summation starts at \(a+1\). Hence,

\begin{equation} \label{eq3}
\begin{split}
\frac{\partial}{\partial t}M_{N(t)}(\theta, t) &= - \sum_{n=a}^{\infty} n e^{n \theta} \lambda P_n(t) + \sum_{n=a+1}^{\infty}(n-1)e^{n \theta}\lambda P_{n-1}(t) \\
&= - \sum_{n=a}^{\infty} n e^{n \theta} \lambda P_n(t) + \sum_{m=a}^{\infty}me^{(m+1) \theta}\lambda P_{m}(t) \\
&= -\lambda \sum_{n=a}^{\infty}n e^{n\theta}P_n(t) + 
\lambda e^{\theta}\sum_{m=a}^{\infty}m e^{m \theta}P_m(t)\\
&= -\lambda \frac{\partial}{\partial \theta}M_{N(t)}(\theta, t) + \lambda e^{\theta} \frac{\partial}{\partial \theta}M_{N(t)}(\theta, t)\\
&= \lambda (e^{\theta} - 1)\frac{\partial}{\partial \theta}M_{N(t)}(\theta, t)
\end{split}
\end{equation}
\end{frame}

\begin{frame}{Moment generating function of \(N(t)\) (cont.)}
\protect\hypertarget{moment-generating-function-of-nt-cont.-3}{}
\begin{equation} \label{eq4}
\begin{split}
\frac{\partial}{\partial t}M_{N(t)}(\theta, t) - \lambda (e^{\theta} - 1)\frac{\partial}{\partial \theta}M_{N(t)}(\theta, t) = 0.
\end{split}
\end{equation}

\begin{block}{Note:}
\protect\hypertarget{note}{}
A partial differential equation (PDE) for a function \(z(x, y)\) is
Lagrange type if it takes the form (General form of first-order
quasilinear PDE)

\begin{equation} \label{eq5}
\begin{split}
P(x, y, z)\frac{\partial z}{\partial x} + Q(x, y, z) \frac{\partial z}{\partial y} = R(x, y, z).
\end{split}
\end{equation}

The associated characteristic system of ordinary differential equations.
\end{block}
\end{frame}

\begin{frame}
\begin{block}{Note (cont)}
\protect\hypertarget{note-cont}{}
\begin{equation} \label{eq5}
\frac{dx}{P(x, y, z)} = \frac{dy}{Q(x, y, z)}=\frac{dz}{R(x, y, z)}.
\end{equation}

is known as the characteristic (auxiliary) system of equation (5).
Suppose that two independent particular solutions of this system have
been found in the form

\(u(x, y, z) = C_1\) and \(v(x, y, z) = C_2\), where where \(C_1\) and
\(C_2\) are arbitrary constants.

Then the general solution to equation (5) can be written as

\begin{equation} \label{eq6}
\phi(u,v)=0
\end{equation}

where \(\phi\) is an arbitrary function of two variables.
\end{block}
\end{frame}

\begin{frame}
\begin{block}{Note (cont.)}
\protect\hypertarget{note-cont.}{}
With equation (6) solved for \(v\), one often specifies the general
solution in the form \(v=\psi(u)\), where \(\psi(u)\) is an arbitrary
function of one variable. The \(\psi\) can be determined using the
boundary conditions.
\end{block}
\end{frame}

\begin{frame}{Moment generating function of \(N(t)\) (cont.)}
\protect\hypertarget{moment-generating-function-of-nt-cont.-4}{}
Revisit equation 4,

\begin{equation}
\begin{split}
\frac{\partial}{\partial \theta}M_{N(t)}(\theta, t) - \lambda (e^{\theta} - 1)\frac{\partial}{\partial \theta}M_{N(t)}(\theta, t) = 0.
\end{split}
\end{equation}

According to the auxiliary system of equation in (6),

\[\frac{dt}{1}=\frac{d\theta}{-\lambda(e^{\theta}-1)}=\frac{M_{N(t)}}{0}\]

\[\frac{dt}{1} = \frac{dM_{N(t)}}{0}\]

\(\frac{dM_{N(t)}}{dt}=0\) \(\Rightarrow\)
\(M_{N(t)}(\theta, t) = constant.\)
\end{frame}

\begin{frame}{Moment generating function of \(N(t)\) (cont.)}
\protect\hypertarget{moment-generating-function-of-nt-cont.-5}{}
Furthermore consider,

\[\frac{dt}{1} = \frac{d\theta}{-\lambda(e^{\theta}-1)}\]

\begin{equation}
\begin{split}
\lambda dt &= -\frac{1}{(e^{\theta}-1)} d\theta \\
&= \frac{-e^{-\theta}}{1-e^{-\theta}}d\theta 
\end{split}
\end{equation}

From equation (9) we can write

\[\lambda t = -ln(1-e^{-\theta}) + c\],
\end{frame}

\begin{frame}{Moment generating function of \(N(t)\) (cont.)}
\protect\hypertarget{moment-generating-function-of-nt-cont.-6}{}
Furthermore

\[ln (e^{\lambda t}) + ln(1-e^{-\theta}) = c.\]

Hence,

\[e^{\lambda t}(1-e^{-\theta}) = constant.\]

Hence, the general solution for eq(8) is

\[M_{N(t)}(\theta, t)= \Psi [e^{\lambda t}(1-e^{-\theta})].\]
\end{frame}

\begin{frame}{Moment generating function of \(N(t)\) (cont.)}
\protect\hypertarget{moment-generating-function-of-nt-cont.-7}{}
The boundary conditions \(P_a(0)=1\), and \(P_n(0)\) for \(n \neq a\),
imply that
\(M_{N(t)}(\theta, 0) = \sum_{n=a}^{\infty} e^{n \theta}P_n(0)=e^{a\theta}\),

\[M_{N(t)}(\theta, 0)= e^{a \theta} = \Psi(1-e^{-\theta}).\]

Let \(\alpha=1-e^{-\theta}\). Then, \(e^{\theta}=(1-\alpha)^{-1}.\)
Hence, \[e^{a\theta} = \Psi(\alpha) = (1-\alpha)^{-a}.\]
\end{frame}

\begin{frame}{Moment generating function of \(N(t)\) (cont.)}
\protect\hypertarget{moment-generating-function-of-nt-cont.-8}{}
Therefore,

\[M_{N(t)}(\theta, t) = \Psi [e^{\lambda t}(1-e^{-\theta})] = [1-e^{\lambda t}(1-e^{-\theta})]^{-a}.\]

Let \(p = e^{-\lambda t}\) and \(p+q=1\). Then,

\[M_{N(t)}(\theta, t) =  [1-p^{-1}(1-e^{-\theta})]^{-a} = \left[ \frac{p-1+e^{-\theta}}{p}\right]^{-a} = \left(\frac{p}{e^{-\theta}-q}\right)^a.\]

Now from this MGF, we can derive the moments of \(N(t)\).
\end{frame}

\begin{frame}{Moment generating function of \(N(t)\) (cont.)}
\protect\hypertarget{moment-generating-function-of-nt-cont.-9}{}
It can be shown that

\(E(N(t)) = a/p = ae^{\lambda t}\) and

\(V[N(t)] = a(1-p)/p^2= a(1-e^{-\lambda t})e^{2\lambda t}.\)

Furthermore, we recognize the above MGF is in the form of the MGF of a
negative binomial random variable \(Y\), with probability mass function
\(P(Y=y) = ^{y-1}C_{a-1}p^{a-1}q^{y-1-(a-1)}p = ^{y-1}C_{a-1}p^aq^{y-a}, \text { }\)
for \(\text{   } y = a, a+1, ...\)

Hence,

\(P(N(t)=n) = ^{n-1}C_{a-1}p^{a}q^{n-a} = ^{n-1}C_{a-1}e^{-\lambda t a}(1-e^{-\lambda t})^{n-a}\)
for \(\text{   } n = a, a+1, ...\)

\begin{longtable}[]{@{}
  >{\raggedright\arraybackslash}p{(\columnwidth - 0\tabcolsep) * \real{0.06}}@{}}
\toprule
\endhead
\#\# Moment generating function of \(N(t)\) - Alternative approach \\
Using probability generating functions. Let \(G_{N(t)}(s, t)\) is called
the probability generating function, \\
\(G_{N(t)}(s, t) = E(s^{N(t)}) = \sum_{n=0}^{\infty}s^nP_n(t).\) \\
The coefficients of \(s^n\) of the expansion of \(G_{N(t)}(s, t)\) will
give \(P_n(t)\). \\
\bottomrule
\end{longtable}
\end{frame}

\begin{frame}{Linear birth process (Yule-Furry Process)}
\protect\hypertarget{linear-birth-process-yule-furry-process-2}{}
\textbf{Summary:}

\begin{block}{When, \(N(0) = 1\)}
\protect\hypertarget{when-n0-1}{}
\vspace{0.5cm}

\(P(N(t)=0)=0\)

\(P(N(t)=n)=e^{-\lambda t}(1-e^{-\lambda t})^{n-1}\text{ }, n \geq 1.\)

\vspace{0.5cm}
\end{block}

\begin{block}{When, \(N(0) = a\)}
\protect\hypertarget{when-n0-a}{}
\vspace{0.5cm}

\(P(N(t)=n) = ^{n-1}C_{a-1}p^{a}q^{n-a} = ^{n-1}C_{a-1}e^{-\lambda t a}(1-e^{-\lambda t})^{n-a}\)
for \(\text{   } n = a, a+1, ...\)
\end{block}
\end{frame}

\begin{frame}{Exercise}
\protect\hypertarget{exercise}{}
Consider a pure birth process on the states \(\{0, 1, ..., N\}\) for
which \(\lambda_k = (N-k) \lambda\) for \(k = 0, 1, ..., N\). Suppose
\(N(0) = 0\). Find \(Pn(t) =P(X(t) = n )\) for \(n = 0, 1\) and 2.
\end{frame}




\end{document}
